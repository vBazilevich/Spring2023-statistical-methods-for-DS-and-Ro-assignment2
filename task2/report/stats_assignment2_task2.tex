\documentclass{article}

\usepackage[a4paper, total={16cm, 26cm}]{geometry}
\usepackage{indentfirst}
\usepackage{graphicx}
\graphicspath{{../images/}}
\usepackage{float}

\usepackage{amsmath}
\usepackage{hyperref}
\usepackage{xcolor}

\title{Statistical techniques for DS and Ro. Assignment 2. Task 2}
\author{Vladimir Bazilevich, v.bazilevich@innopolis.university}

\setcounter{section}{0}

\begin{document}
	\maketitle
	
	\href{https://github.com/vBazilevich/Spring2023-statistical-methods-for-DS-and-Ro-assignment2}{Assignment 2 repository}.
	
	\href{https://github.com/vBazilevich/Spring2023-statistical-methods-for-DS-and-Ro-assignment2/tree/master/task2}{Task 2 folder}.
	
	\section{Effect of different schedules}
	For my experiments I have decided to update the temperature every 5 trials. I have selected different decay rates: $[0.3, 0.9, 0.99. 0.999]$
	
	\begin{figure}[H]
		\includegraphics[scale=0.8]{sa\_td0.3\_ur5}
		\caption{Simulated annealing with temperature decay rate $\alpha=0.3$}
		\centering
	\end{figure}

	\begin{figure}[H]
		\includegraphics[scale=0.8]{sa\_td0.9\_ur5}
		\caption{Simulated annealing with temperature decay rate $\alpha=0.9$}
		\centering
	\end{figure}

	\begin{figure}[H]
		\includegraphics[scale=0.8]{sa\_td0.99\_ur5}
		\caption{Simulated annealing with temperature decay rate $\alpha=0.99$}
		\centering
	\end{figure}

	\begin{figure}[H]
		\includegraphics[scale=0.8]{sa\_td0.999\_ur5}
		\caption{Simulated annealing with temperature decay rate $\alpha=0.999$}
		\centering
	\end{figure}

	We can observe several effects from pictures above:
	\begin{enumerate}
		\item Cool down time increases as decay rate increases
		\item High decay rates tend to produce better final results
		\item Exploration intensity decreases over time
	\end{enumerate}

	Those trends also appeared after multiple restarts.
\end{document}